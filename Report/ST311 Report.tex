% First we set up the style for the dissertation. Do not change this!
\documentclass[a4paper,oneside,11pt]{report}

% Now we load the preamble, which sets up most we need for the dissertation.
\usepackage[left=4cm,right=2.9cm,top=3cm,bottom=3cm]{geometry}
\linespread{1.25}

% Next we load some more packages. You might want to add to this list, but
% do not remove or change things unless you know what you are doing.

\usepackage{fancyhdr}
\pagestyle{fancy}
\rhead{Page~\thepage} % Page number at the top right.
\cfoot{} % No page number at the bottom.
\usepackage{graphics}
\usepackage{amsmath,amsfonts,amssymb,amsthm}
\usepackage{epsfig,epstopdf,color}
\usepackage{enumitem}
\usepackage[nottoc]{tocbibind}

% Note that we set some options for algorithm2e here. If you want another
% look to the result, change these.
\usepackage[english,algoruled,lined]{algorithm2e}

\usepackage{tikz}
\usetikzlibrary{arrows,automata}

% \usepackage[notcite,notref]{showkeys}
% Un-comment the line above this to see keys in the compiled output.

% Here is the place to put any more packages you need.

% The hyperref package is special: it has to be last.
\usepackage[hidelinks,breaklinks]{hyperref}

% Now we define a lot of commands and environments. The next five lines are
% options for enumerate.
\newcommand{\itm}[1]{\textrm{({#1})}} 
\newcommand{\itmit}[1]{\itm{\textit{#1\,}}} 
\newcommand{\rom}{\itmit{\roman{*}}} 
\newcommand{\abc}{\itmit{\alph{*}}}
\newcommand{\arab}{\itmit{\arabic{*}}} 

% The next line sets the spacing between items in itemize and enumerate
% lists.
\setlist{itemsep=3pt,parsep=0pt,topsep=2pt,partopsep=0pt}  

%The next 10 lines set up the claimproof environment.
\newcommand{\oldqed}{}
\newcommand{\eofClaim}{\hfill\scalebox{.6}{$\Box$}}
\newenvironment{claimproof}[1][Proof]{
  \renewcommand{\oldqed}{\qedsymbol}
  \renewcommand{\qedsymbol}{\eofClaim}
  \begin{proof}[#1]
  }{
  \end{proof}
  \renewcommand{\qedsymbol}{\oldqed}
}

% The next lines allow explanations above <,>,=,\le,\ge 
\newcommand{\By}[2]{\overset{\mbox{\tiny{#1}}}{#2}} 
\newcommand{\ByRef}[2]{   \By{\eqref{#1}}{#2} } 
\newcommand{\eqBy}[1]{    \By{#1}{=} } 
\newcommand{\lBy}[1]{     \By{#1}{<} } 
\newcommand{\gBy}[1]{     \By{#1}{>} } 
\newcommand{\leBy}[1]{    \By{#1}{\le} } 
\newcommand{\geBy}[1]{    \By{#1}{\ge} } 
\newcommand{\eqByRef}[1]{ \ByRef{#1}{=} } 
\newcommand{\lByRef}[1]{  \ByRef{#1}{<} } 
\newcommand{\gByRef}[1]{  \ByRef{#1}{>} } 
\newcommand{\leByRef}[1]{ \ByRef{#1}{\le} } 
\newcommand{\geByRef}[1]{ \ByRef{#1}{\ge} }

% The next two lines typeset \le and \ge nicely.
\renewcommand{\le}{\leqslant}
\renewcommand{\ge}{\geqslant} 

%Now we define some 'theorem like' environments.
\newtheorem{theorem}{Theorem}
\newtheorem{lemma}[theorem]{Lemma}
\newtheorem{claim}[theorem]{Claim}
\newtheorem{corollary}[theorem]{Corollary}
\newtheorem{conjecture}[theorem]{Conjecture}
\theoremstyle{definition}
\newtheorem{definition}[theorem]{Definition}

% The next lines define the title and candidate number.
\newcommand{\dissertationtitle}[1]{\newcommand*{\DissertationTitle}{#1}}

\newcommand{\candnumber}[1]{\newcommand*{\CandNumber}{#1}}

% The next bit sets up the title page.

\newcommand{\dissertationtitlepage}{%
  \pagenumbering{roman}
  \thispagestyle{empty}
  \vspace*{20mm}
  \begin{center}
    \Huge\textbf{\textsf{\DissertationTitle}}
  \end{center}

  \vspace{40mm}
  \mbox{}

  \vfill
  \begin{center}
    \Large\textsf{Candidate Number: \CandNumber}
  \end{center}

  \vspace{25mm}
  \mbox{}

  \vfill
  \begin{center}
    \Large\textbf{\textsf{A Dissertation submitted to the Department of
        Mathematics\\
        of the London School of Economics and Political Science}}
  \end{center}

  \vfill
  \begin{center}
    \Large\textsf{\today}
  \end{center}}

% Next the environment for the summary.

\newenvironment{dissertationsummary}{%
  \chapter*{Summary}
  \addcontentsline{toc}{chapter}{Summary}
  \lhead{Summary}}{%
  \clearpage
  \lhead{\nouppercase\leftmark}}

% Table of contents.

\newcommand{\dissertationtableofcontents}{%
  \tableofcontents\clearpage\pagenumbering{arabic}}





% The next lines set the title and candidate number. Fill in your own!
\dissertationtitle{Template for\\
  the Dissertation in Mathematics}

\candnumber{007}

% Now the document begings.

\begin{document}

% First the titlepage.

\dissertationtitlepage

% You are required to write a short summary.

\begin{dissertationsummary}
  This is a template file for your dissertation, matching the regulations,
  with this summary, a table of contents, an introduction and a conclusion,
  which are all compulsory parts of your dissertation. You do not have to
  use this template file, but you may find it useful. You are free to
  change how things look, but keep things like the width and height of the
  text, the space between lines and the font size more or less the same.

  To compile it, you need to run \texttt{pdflatex} (not \texttt{latex}) and
  \texttt{bibtex}. You may need to run \texttt{pdflatex} at least twice, to
  get all the references, table of contents, etc.\ correct. If you are on
  a School computer, it should work without problems. If you are using your
  own computer and it does not work, make sure your \TeX\ and \LaTeX\
  distribution are up-to-date.

  Further information about the requirements and expectations of the
  Dissertation can be found in the \textit{Instructions and Guidelines for
    the Dissertation in Applicable Mathematics}~\cite{IandG1718}.
\end{dissertationsummary}

% Next we set up the table of contents page. This will update
% automagically, so you do not need to change anything. You may need to run
% pdflatex two or three times before it all updates correctly though.

\dissertationtableofcontents

% Finally, we can get started 

\chapter{Introduction}

\chapter{Another chapter}

\chapter{And another one \ldots}

\chapter{And another one \ldots}

\chapter{And another one \ldots}

But don't overdo the number of chapters.

\chapter{Conclusion}

The end

% Time to add the bibliograph

\bibliographystyle{plain}
\bibliography{template-BibFile}

% If you have one or more appendices, this is where they should go. 
% The next command changes the normal \chapter{...} command.

\appendix

% So to get one or more appendices, you do this.

\chapter{The first appendix}

\chapter{A second appendix}

Again, don't overdo the number of appendices.

\end{document}
