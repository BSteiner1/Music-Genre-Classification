\usepackage[left=4cm,right=2.9cm,top=3cm,bottom=3cm]{geometry}
\linespread{1.25}

% Next we load some more packages. You might want to add to this list, but
% do not remove or change things unless you know what you are doing.

\usepackage{fancyhdr}
\pagestyle{fancy}
\rhead{Page~\thepage} % Page number at the top right.
\cfoot{} % No page number at the bottom.
\usepackage{graphics}
\usepackage{amsmath,amsfonts,amssymb,amsthm}
\usepackage{epsfig,epstopdf,color}
\usepackage{enumitem}
\usepackage[nottoc]{tocbibind}

% Note that we set some options for algorithm2e here. If you want another
% look to the result, change these.
\usepackage[english,algoruled,lined]{algorithm2e}

\usepackage{tikz}
\usetikzlibrary{arrows,automata}

% \usepackage[notcite,notref]{showkeys}
% Un-comment the line above this to see keys in the compiled output.

% Here is the place to put any more packages you need.

% The hyperref package is special: it has to be last.
\usepackage[hidelinks,breaklinks]{hyperref}

% Now we define a lot of commands and environments. The next five lines are
% options for enumerate.
\newcommand{\itm}[1]{\textrm{({#1})}} 
\newcommand{\itmit}[1]{\itm{\textit{#1\,}}} 
\newcommand{\rom}{\itmit{\roman{*}}} 
\newcommand{\abc}{\itmit{\alph{*}}}
\newcommand{\arab}{\itmit{\arabic{*}}} 

% The next line sets the spacing between items in itemize and enumerate
% lists.
\setlist{itemsep=3pt,parsep=0pt,topsep=2pt,partopsep=0pt}  

%The next 10 lines set up the claimproof environment.
\newcommand{\oldqed}{}
\newcommand{\eofClaim}{\hfill\scalebox{.6}{$\Box$}}
\newenvironment{claimproof}[1][Proof]{
  \renewcommand{\oldqed}{\qedsymbol}
  \renewcommand{\qedsymbol}{\eofClaim}
  \begin{proof}[#1]
  }{
  \end{proof}
  \renewcommand{\qedsymbol}{\oldqed}
}

% The next lines allow explanations above <,>,=,\le,\ge 
\newcommand{\By}[2]{\overset{\mbox{\tiny{#1}}}{#2}} 
\newcommand{\ByRef}[2]{   \By{\eqref{#1}}{#2} } 
\newcommand{\eqBy}[1]{    \By{#1}{=} } 
\newcommand{\lBy}[1]{     \By{#1}{<} } 
\newcommand{\gBy}[1]{     \By{#1}{>} } 
\newcommand{\leBy}[1]{    \By{#1}{\le} } 
\newcommand{\geBy}[1]{    \By{#1}{\ge} } 
\newcommand{\eqByRef}[1]{ \ByRef{#1}{=} } 
\newcommand{\lByRef}[1]{  \ByRef{#1}{<} } 
\newcommand{\gByRef}[1]{  \ByRef{#1}{>} } 
\newcommand{\leByRef}[1]{ \ByRef{#1}{\le} } 
\newcommand{\geByRef}[1]{ \ByRef{#1}{\ge} }

% The next two lines typeset \le and \ge nicely.
\renewcommand{\le}{\leqslant}
\renewcommand{\ge}{\geqslant} 

%Now we define some 'theorem like' environments.
\newtheorem{theorem}{Theorem}
\newtheorem{lemma}[theorem]{Lemma}
\newtheorem{claim}[theorem]{Claim}
\newtheorem{corollary}[theorem]{Corollary}
\newtheorem{conjecture}[theorem]{Conjecture}
\theoremstyle{definition}
\newtheorem{definition}[theorem]{Definition}

% The next lines define the title and candidate number.
\newcommand{\dissertationtitle}[1]{\newcommand*{\DissertationTitle}{#1}}

\newcommand{\candnumber}[1]{\newcommand*{\CandNumber}{#1}}

% The next bit sets up the title page.

\newcommand{\dissertationtitlepage}{%
  \pagenumbering{roman}
  \thispagestyle{empty}
  \vspace*{20mm}
  \begin{center}
    \Huge\textbf{\textsf{\DissertationTitle}}
  \end{center}

  \vspace{40mm}
  \mbox{}

  \vfill
  \begin{center}
    \Large\textsf{Candidate Number: \CandNumber}
  \end{center}

  \vspace{25mm}
  \mbox{}

  \vfill
  \begin{center}
    \Large\textbf{\textsf{A Dissertation submitted to the Department of
        Mathematics\\
        of the London School of Economics and Political Science}}
  \end{center}

  \vfill
  \begin{center}
    \Large\textsf{\today}
  \end{center}}

% Next the environment for the summary.

\newenvironment{dissertationsummary}{%
  \chapter*{Summary}
  \addcontentsline{toc}{chapter}{Summary}
  \lhead{Summary}}{%
  \clearpage
  \lhead{\nouppercase\leftmark}}

% Table of contents.

\newcommand{\dissertationtableofcontents}{%
  \tableofcontents\clearpage\pagenumbering{arabic}}



